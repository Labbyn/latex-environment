\chapter{Wymagania na system}

%----------------------------------------------------------------------------------------------------
%---------------------------------------OGÓLNE-I-DZIEDZINOWE-----------------------------------------
%----------------------------------------------------------------------------------------------------

\section{Wymagania ogólne i dziedzinowe}
% LAB-D-01
\begin{requirementstab}
    \id{LAB-D-01}
    \priority{M}
    \name{Kontrola dostępu}
    \descr{System musi być dostępny tylko dla \gls{użytkownik}ów posiadających konto w systemie}
    \sholder{Zespół projektowy}
    \reqrelated{Brak}
\end{requirementstab}

% LAB-D-02
\begin{requirementstab}
    \id{LAB-D-02}
    \priority{M}
    \name{Zarządzanie \gls{laboratorium}}
    \descr{System musi wspierać zarządzanie \gls{laboratorium} IT}
    \sholder{Zespół projektowy}
    \reqrelated{Brak}
\end{requirementstab}

% LAB-D-03
\begin{requirementstab}
    \id{LAB-D-03}
    \priority{M}
    \name{Inwentarz Techniczny}
    \descr{System musi wspierać zarządzanie inwentarzem technicznym}
    \sholder{Zespół projektowy}
    \reqrelated{Brak}
\end{requirementstab}

% LAB-D-04
\begin{requirementstab}
    \id{LAB-D-04}
    \priority{M}
    \name{Monitorowanie stanu}
    \descr{System musi wspierać monitorowanie stanu urządzeń}
    \sholder{Zespół projektowy}
    \reqrelated{Brak}
\end{requirementstab}

% LAB-D-05
\begin{requirementstab}
    \id{LAB-D-05}
    \priority{M}
    \name{Mapa przestrzenna \gls{laboratorium}}
    \descr{System musi umożliwiać odwzorowanie fizycznego układu \gls{laboratorium}}
    \sholder{Zespół projektowy}
    \reqrelated{Brak}
\end{requirementstab}

% LAB-D-06
\begin{requirementstab}
    \id{LAB-D-06}
    \priority{M}
    \name{Ochrona danych}
    \descr{System musi spełniać podstawowe wymogi ochrony danych}
    \sholder{Zespół projektowy}
    \reqrelated{Brak}
\end{requirementstab}

%----------------------------------------------------------------------------------------------------
%----------------------------------------FUNKCJONALNE------------------------------------------------
%----------------------------------------------------------------------------------------------------

\section{Wymagania Funkcjonalne}

\subsection{Zakładka "Labs"}

% LAB-F-01
\begin{requirementstab}
    \id{LAB-F-01}
    \priority{M}
    \name{System musi posiadać widok całego \gls{laboratorium}}
    \descr{Jako pracownik związany ze sprzętem w \gls{laboratorium}/serwerowni chcę mieć możliwość
sprawdzenia układu pomieszczenia w celu efektownego planowania przyszłych i bieżących zadań}
    \acceptcrit{System posiada w pełni funkcjonalną zakładkę „Map” dla każdego \gls{laboratorium}}
    \inputdata{Lab z mapą przestrzenną, \gls{użytkownik} systemu z dowolną przypisaną rolą}
    \preconditions{\gls{użytkownik} jest zalogowany do systemu i widzi zakładkę „Labs”, a wewnątrz niej przynajmniej 1 Lab}
    \postconditions{Mapa przestrzenna \gls{laboratorium} jest widoczna i w pełni funkcjonalna}
    \exceptions{Lab nie posiada mapy \gls{laboratorium}}
    \implementation{Dodanie przycisku „Show Map” do każdego obiektu \gls{laboratorium}.}
    \sholder{Zespół projektowy}
    \reqrelated{LAB-F-14}
\end{requirementstab}

% LAB-F-14
\begin{requirementstab}
    \id{LAB-F-14}
    \priority{M}
    \name{System musi posiadać system Drag n' Drop elementów \gls{laboratorium}}
    \descr{Jako \gls{użytkownik} systemu, chcę móc ustawić elementy interaktywne na mapie przestrzennej \gls{laboratorium}, żeby odwzorować stan faktyczny pomieszczenia.}
    \acceptcrit{Mapa przestrzenna \gls{laboratorium} posiada w pełni funkcjonalny system Drag n’ Drop z możliwością aranżacji elementów na dostępnej powierzchni.}
    \inputdata{Lab z mapą przestrzenną, \gls{użytkownik} systemu z dowolną przypisaną rolą z uprawnieniami do wybranego \gls{laboratorium}}
    \preconditions{\gls{użytkownik} jest zalogowany do systemu, w zakładce „Labs” widzi przynajmniej 1 lab do którego ma uprawnienia edycji z przyciskiem „Show Map”, po kliknięciu na ten przycisk przeniesiony jest do widoku przestrzennego \gls{laboratorium}.}
    \postconditions{\gls{użytkownik} jest w stanie rozmieszczać elementy \gls{laboratorium} w dowolny sposób z możliwością zapisu zmian.}
    \exceptions{\gls{użytkownik} nie posiada praw edycji \gls{laboratorium} na mapie przestrzennej.}
    \implementation{Dodanie przycisku „Edit” możliwego do kliknięcia w momencie gdy zalogowany \gls{użytkownik} jest przypisany do danego \gls{laboratorium}.}
    \sholder{Zespół projektowy}
    \reqrelated{LAB-F-01}
\end{requirementstab}

\subsection{Panel Urządzenia}

% LAB-F-02
\begin{requirementstab}
    \id{LAB-F-02}
    \priority{M}
    \name{System musi pokazywać statystyki urządzeń}
    \descr{Jako \gls{użytkownik} systemu, chcę mieć możliwość odczytania statystyk każdego urządzenia w jego unikalnej zakładce z informacjami}
    \acceptcrit{Każde urządzenie posiada widok ze szczegółowymi statystykami}
    \inputdata{System z przynajmniej 1 dostępnym urządzeniem, \gls{użytkownik} systemu z dowolną przypisaną rolą}
    \preconditions{\gls{użytkownik} jest zalogowany do systemu, w zakładce „Labs” widzi przynajmniej 1 \gls{laboratorium} z przypisanym przynajmniej 1 urządzeniem, który może kliknąć}
    \postconditions{\gls{użytkownik} jest w stanie przenieść się na widok ze szczegółami urządzenia po kliknięciu kafelka z nazwą urządzenia}
    \exceptions{Brak dostępnych urządzeń w systemie}
    \implementation{Po kliknięciu na obiekt maszyny, aktywowanie okna urządzenia ze statystykami}
    \sholder{Zespół projektowy}
    \reqrelated{LAB-F-03, LAB-F-05, LAB-F-10, LAB-F-11, LAB-F-15}
\end{requirementstab}

% LAB-F-03
\begin{requirementstab}
    \id{LAB-F-03}
    \priority{C}
    \name{System powinien pozwolić na wykonywanie zdalnych akcji np. Reboot, ping}
    \descr{Jak \gls{użytkownik} systemu chcę mieć dostęp do wykonywania zdalnych komend na maszynie z poziomu widoku maszyny}
    \acceptcrit{System posiada możliwość wykonania zdalnej komendy z poziomu interfejsu \gls{użytkownik}a}
    \inputdata{System z przynajmniej 1 dostępnym urządzeniem, \gls{użytkownik} systemu z dowolną przypisaną rolą}
    \preconditions{\gls{użytkownik} jest zalogowany do systemu, na ekranie szczegółów urządzenia widoczny jest panel do wpisania zdalnej komendy}
    \postconditions{\gls{użytkownik} jest w stanie wysłać komendę zdalnie na urządzenie z poziomu systemu}
    \exceptions{Wybrane urządzenie nie jest podłączone do sieci}
    \implementation{Na ekranie szczegółów urządzenia dodanie sekcji do wysyłania zdalnego polecenia.}
    \sholder{Zespół projektowy}
    \reqrelated{LAB-F-02, LAB-F-05, LAB-F-10, LAB-F-11, LAB-F-15}
\end{requirementstab}

% LAB-F-05
\begin{requirementstab}
    \id{LAB-F-05}
    \priority{M}
    \name{System musi zezwolić na przypisywanie platform do konkretnych \gls{rack}ów i półek}
    \descr{Jako \gls{użytkownik} chcę mieć możliwość przypisania platformy do wybranej przeze mnie półki w celu skutecznego wyszukiwania urządzeń znajdujących się na tej samej półce lub \gls{rack}u.}
    \acceptcrit{System posiada możliwość przypisania platformy do jednego z \gls{rack}ów lub półek.}
    \inputdata{System z przynajmniej 1 dostępnym urządzeniem, labem oraz półką lub \gls{rack}iem, \gls{użytkownik} systemu z dowolną przypisaną rolą z przynajmniej 1 przypisaną do niego lub jego grupy maszyną}
    \preconditions{\gls{użytkownik} jest zalogowany do systemu, na ekranie szczegółów urządzenia widoczny jest aktualnie przypisany \gls{rack} lub półka}
    \postconditions{\gls{użytkownik} jest w stanie zmienić przypisanie urządzenia do innego recku lub półki}
    \exceptions{Brak innych \gls{rack}ów lub półek w systemie}
    \implementation{Na ekranie szczegółów urządzenia dodanie sekcji z przypisanym \gls{rack}iem lub półką}
    \sholder{Zespół projektowy}
    \reqrelated{LAB-F-02, LAB-F-03, LAB-F-10, LAB-F-11, LAB-F-15}
\end{requirementstab}

% LAB-F-10
\begin{requirementstab}
    \id{LAB-F-10}
    \priority{W}
    \name{System powinien posiadać odnośnik do informacji o danym sprzęcie/ layout'u PCB}
    \descr{Jako \gls{użytkownik} chcę mieć możliwość przejść do karty produktu ze szczegółami technicznymi lub dotyczącymi PCB z poziomu systemu}
    \acceptcrit{System posiada odnośnik do strony z dokumentacją techniczną sprzętu}
    \inputdata{System z przynajmniej 1 dostępnym urządzeniem, \gls{użytkownik} systemu z dowolną przypisaną rolą}
    \preconditions{Brak, wymaganie odrzucone}
    \postconditions{Brak, wymaganie odrzucone}
    \exceptions{Brak, wymaganie odrzucone}
    \implementation{Brak, wymaganie odrzucone}
    \sholder{Zespół projektowy}
    \reqrelated{LAB-F-02, LAB-F-03, LAB-F-05, LAB-F-11, LAB-F-15}
\end{requirementstab}

% LAB-F-11
\begin{requirementstab}
    \id{LAB-F-11}
    \priority{M}
    \name{Dane w widoku urządzenia}
    \descr{Jako \gls{użytkownik} systemu, chcę mieć dostęp do następujących danych na poziomie widoku urządzenia: Lokalizacja, Nazwa Hosta, Adres IP oraz MAC, Status w sieci, Port w \gls{PDU}, Tag (przypisany zespół / przeznaczony zespół), System Operacyjny, Informacje o sprzęcie (CPU, Dysk, RAM), Numer Seryjny}
    \acceptcrit{Wszystkie pola zawarte w wymaganiu są dostępne w systemie w widoku urządzenia}
    \inputdata{System z przynajmniej 1 dostępnym urządzeniem, \gls{użytkownik} systemu z dowolną przypisaną rolą z przynajmniej 1 przypisaną do niego lub jego grupy maszyną}
    \preconditions{\gls{użytkownik} jest zalogowany do systemu, na ekranie szczegółów urządzenia}
    \postconditions{\gls{użytkownik} jest w stanie zobaczyć wszystkie dane lub miejsce na ich umieszczenie w systemie na poziomie widoku urządzenia}
    \exceptions{Brak dostępnych lub przypisanych do \gls{użytkownik}a urządzeń w systemie}
    \implementation{Na ekranie szczegółów urządzenia dodać pola wypisane w wymaganiu.}
    \sholder{Zespół projektowy}
    \reqrelated{LAB-F-02, LAB-F-03, LAB-F-05, LAB-F-10, LAB-F-15}
\end{requirementstab}

% LAB-F-15
\begin{requirementstab}
    \id{LAB-F-15}
    \priority{M}
    \name{System musi posiadać widok urządzenia}
    \descr{Jako \gls{użytkownik} systemu chcę mieć dostęp do szczegółów urządzenia z poziomu systemu}
    \acceptcrit{Zaimplementowany panel szczegółów urządzenia}
    \inputdata{System z przynajmniej 1 dostępnym urządzeniem, \gls{użytkownik} systemu z dowolną przypisaną rolą z przynajmniej 1 przypisaną do niego lub jego grupy maszyną}
    \preconditions{\gls{użytkownik} jest zalogowany do systemu}
    \postconditions{\gls{użytkownik} jest w stanie przejść do widoku ze szczegółami urządzenia z poziomu systemu}
    \exceptions{Brak dostępnych urządzeń w systemie}
    \implementation{Dodanie widoku urządzenia po kliknięciu w kafelek z nazwą urządzenia z poziomu mapy przestrzennej lub paska wyszukiwania}
    \sholder{Zespół projektowy}
    \reqrelated{LAB-F-02, LAB-F-03, LAB-F-05, LAB-F-10, LAB-F-11}
\end{requirementstab}

\subsection{Zakładka "Inventory"}

% LAB-F-04
\begin{requirementstab}
    \id{LAB-F-04}
    \priority{S}
    \name{System powinien posiadać inwentarz materiałów eksploatacyjnych}
    \descr{Jako \gls{użytkownik} systemu chcę mieć dostęp do inwentarza materiałów eksploatacyjnych aby szybko i efektywnie wyszukiwać niezbędne do mojej pracy narzędzia i kontrolować ich ilość}
    \acceptcrit{Zakładka „Inventory” zaimplementowana w systemie}
    \inputdata{\gls{użytkownik} systemu z dowolną przypisaną rolą}
    \preconditions{\gls{użytkownik} jest zalogowany do systemu}
    \postconditions{\gls{użytkownik} widzi w pasku bocznym zakładkę „Inventory”, po której kliknięciu przenoszony jest na ekran inwentarza}
    \exceptions{Brak paska bocznego}
    \implementation{Dodanie zakładki „Inventory” w paski bocznym}
    \sholder{Zespół projektowy}
    \reqrelated{LAB-F-12}
\end{requirementstab}

% LAB-F-12
\begin{requirementstab}
    \id{LAB-F-12}
    \priority{M}
    \name{Informacje w inwentarzu technicznym}
    \descr{Jako \gls{użytkownik} systemu, chcę mieć dostęp do następujących danych na poziomie inwentarza technicznego: Nazwa i nr seryjny, Dostępność, przypisanie do konkretnych maszyn, przypisanie do konkretnych zespołów, typ urządzenia/kategorie, Historia wypożyczeń, lokalizacja}
    \acceptcrit{Wszystkie pola zawarte w wymaganiu są dostępne w systemie w inwentarzu technicznym}
    \inputdata{\gls{użytkownik} systemu z dowolną przypisaną rolą, przynajmniej 1 element w inwentarzu przypisany do \gls{użytkownik}a lub jego grupy}
    \preconditions{\gls{użytkownik} zalogowany do systemu na zakładce „Inventory”}
    \postconditions{\gls{użytkownik} jest w stanie zobaczyć wszystkie dane lub miejsce na ich umieszczenie w systemie na poziomie inwentarzu}
    \exceptions{Brak przypisanych elementów do \gls{użytkownik}a w inwentarzu}
    \implementation{Do każdego elementu inwentarzu dodać pola wypisane w wymaganiu.}
    \sholder{Zespół projektowy}
    \reqrelated{LAB-F-04}
\end{requirementstab}

\subsection{Zakładka "Dashboard"}

% LAB-F-06
\begin{requirementstab}
    \id{LAB-F-06}
    \priority{S}
    \name{System powinien posiadać panel z podsumowaniem informacji o sprzęcie przypisanym do \gls{użytkownik}a}
    \descr{Jako \gls{użytkownik} systemu chcę mieć dostęp do panelu z podsumowaniem elementów przypisanych do mnie, co pozwoli skrócić czas wyszukiwania go}
    \acceptcrit{Zakładka „Dashboard” widoczna z poziomu systemu na pasku bocznym}
    \inputdata{\gls{użytkownik} systemu z dowolną przypisaną rolą}
    \preconditions{\gls{użytkownik} zalogowany do systemu}
    \postconditions{\gls{użytkownik} widzi zakładkę „Dashboard” na pasku bocznym, po której kliknięciu przenoszony jest na ekran z podsumowaniem}
    \exceptions{\gls{użytkownik} nie ma przypisanego żadnego elementu}
    \implementation{Dodanie zakładki „Dashboard” na pasku bocznym}
    \sholder{Zespół projektowy}
    \reqrelated{Brak}
\end{requirementstab}

\subsection{Search Bar (Pasek wyszukiwania)}

% LAB-F-07
\begin{requirementstab}
    \id{LAB-F-07}
    \priority{M}
    \name{System musi posiadać wyszukiwarkę sprzętu}
    \descr{Jak \gls{użytkownik} systemu chcę mieć dostęp do paska wyszukiwania by przyśpieszyć identyfikowanie sprzętu}
    \acceptcrit{Pasek wyszukiwania zaimplementowany w systemie i dodany do paska bocznego}
    \inputdata{\gls{użytkownik} systemu z dowolną przypisaną rolą}
    \preconditions{\gls{użytkownik} zalogowany do systemu}
    \postconditions{\gls{użytkownik} widzi pasek wyszukiwania na pasku bocznym, w którym jest w stanie wpisać wyszukiwaną frazę}
    \exceptions{Brak}
    \implementation{Dodanie paska wyszukiwania na pasku bocznym}
    \sholder{Zespół projektowy}
    \reqrelated{Brak}
\end{requirementstab}

\subsection{Zakładka "History"}

% LAB-F-08
\begin{requirementstab}
    \id{LAB-F-08}
    \priority{M}
    \name{System musi posiadać historię zmian sprzętu}
    \descr{Jako \gls{użytkownik} chcę śledzić zmiany sprzętu by efektywnie identyfikować modyfikacje w jego konfiguracji}
    \acceptcrit{Mechanizm zapisywania historii akcji zaimplementowany w systemie}
    \inputdata{\gls{użytkownik} systemu z dowolną przypisaną rolą i przynajmniej 1 przypisanym elementem do niego lub jego grupy}
    \preconditions{\gls{użytkownik} zalogowany do systemu na ekranie wybranego elementu}
    \postconditions{\gls{użytkownik} wykonuje zmianę, która jest wykryta przez system}
    \exceptions{Brak}
    \implementation{Dodanie mechanizmu zapisywania wykonywanych akcji w systemie}
    \sholder{Zespół projektowy}
    \reqrelated{LAB-F-16, LAB-F-21}
\end{requirementstab}

% LAB-F-16
\begin{requirementstab}
    \id{LAB-F-16}
    \priority{C}
    \name{System powinien rejestrować działania \gls{użytkownik}ów i zapisywać je do pliku}
    \descr{Jako \gls{użytkownik} systemu chcę by zmiany wykonywane w systemie zapisywane były w pliku by zapewnić trwałość danych w kopiach zapasowych systemu}
    \acceptcrit{Zmiany logowane przez system zapisywane są do pliku}
    \inputdata{\gls{użytkownik} systemu z dowolną przypisaną rolą i przynajmniej 1 przypisanym elementem do niego lub jego grupy}
    \preconditions{\gls{użytkownik} zalogowany do systemu na ekranie wybranego elementu jest w stanie wykonać modyfikację}
    \postconditions{Zmiana wykryta przez system, zapisana do pliku}
    \exceptions{Brak miejsca na przestrzeni dyskowej, na której system domyślnie operuje}
    \implementation{Dodanie funkcjonalności zapisu historii do pliku}
    \sholder{Zespół projektowy}
    \reqrelated{LAB-F-08, LAB-F-21}
\end{requirementstab}

% LAB-F-21
\begin{requirementstab}
    \id{LAB-F-21}
    \priority{C}
    \name{Wersjonowanie historii}
    \descr{Jako \gls{użytkownik} systemu chcę mieć możliwość wersjonowania historii zmian żeby w prosty sposób sprawdzić stan przed modyfikacją}
    \acceptcrit{Funkcjonalność wersjonowania historii zaimplementowana w systemie z możliwością wyboru wersji}
    \inputdata{\gls{użytkownik} systemu z dowolną przypisaną rolą i przynajmniej 1 przypisanym elementem do niego lub jego grupy, na którym wykonane zostały modyfikacje}
    \preconditions{\gls{użytkownik} z poziomu widoku wybranego elementu widzi pasek wyboru z opisem wersji}
    \postconditions{\gls{użytkownik} jest w stanie wybrać wcześniejszą wersję widoku z przed zmian}
    \exceptions{Brak zapisanych zmian w elemencie}
    \implementation{Dodanie funkcjonalności wersjonowania historii }
    \sholder{Zespół projektowy}
    \reqrelated{LAB-F-08, LAB-F-16}
\end{requirementstab}

\subsection{Zakładka "Users"}

% LAB-F-09
\begin{requirementstab}
    \id{LAB-F-09}
    \priority{M}
    \name{System musi posiadać panel zarządzania \gls{użytkownik}ami}
    \descr{Jako \gls{użytkownik} z uprawnieniami administratora systemu chcę mieć dostęp do panelu, z poziomu którego będę w stanie zarządzać \gls{użytkownik}ami systemu}
    \acceptcrit{Zakładka „Users” zaimplementowana w systemie}
    \inputdata{\gls{użytkownik} systemu z przypisaną rolą administratora, dodatkowy \gls{użytkownik} w systemie z dowolną rolą}
    \preconditions{\gls{użytkownik} z rolą administratora zalogowany do systemu}
    \postconditions{\gls{użytkownik} z rolą administratora widzi zakładkę „Users” na pasku bocznym systemu}
    \exceptions{Brak}
    \implementation{Dodanie i implementacja zakładki „Users” w systemie}
    \sholder{Zespół projektowy}
    \reqrelated{LAB-F-17}
\end{requirementstab}

% LAB-F-17
\begin{requirementstab}
    \id{LAB-F-17}
    \priority{M}
    \name{System powinien rozróżniać 2 role \gls{użytkownik}ów: administrator, \gls{użytkownik}}
    \descr{Jako \gls{użytkownik} systemu chcę aby posiadał on kontrolę dostępu w postaci ról \gls{użytkownik}ów: administrator oraz \gls{użytkownik}}
    \acceptcrit{System posiadający role \gls{użytkownik}ów opisane w wymaganiu}
    \inputdata{Brak}
    \preconditions{System posiadający funkcjonalność autoryzacji \gls{użytkownik}ów}
    \postconditions{Możliwość wyboru roli pomiędzy: administrator i \gls{użytkownik}}
    \exceptions{Brak}
    \implementation{Dodanie pola „role” w bazie danych \gls{użytkownik}ów}
    \sholder{Zespół projektowy}
    \reqrelated{LAB-F-09}
\end{requirementstab}

\subsection{Zakładka "Groups"}

% LAB-F-18
\begin{requirementstab}
    \id{LAB-F-18}
    \priority{M}
    \name{System musi zezwalać na przypisywanie \gls{użytkownik}ów do istniejących grup}
    \descr{Jako \gls{użytkownik} systemu z uprawnieniami administratora chcę mieć możliwość przypisania \gls{użytkownik}ów do grup co pozwoli efektywniej zarządzać zespołami w organizacji}
    \acceptcrit{Funkcjonalność przypisywania \gls{użytkownik}a zaimplementowana w systemie}
    \inputdata{System z funkcjonalną bazą danych}
    \preconditions{System z przynajmniej 1 \gls{użytkownik}iem w systemie}
    \postconditions{Możliwość przypisania \gls{użytkownik}a do grupy}
    \exceptions{Brak}
    \implementation{Dodanie pola „group” w bazie danych \gls{użytkownik}ów}
    \sholder{Zespół projektowy}
    \reqrelated{LAB-F-19}
\end{requirementstab}

% LAB-F-19
\begin{requirementstab}
    \id{LAB-F-19}
    \priority{M}
    \name{System musi zezwalać na tworzenie grup \gls{użytkownik}ów}
    \descr{Jako \gls{użytkownik} z uprawnieniami administratora chcę mieć możliwość tworzenia grup \gls{użytkownik}ów z poziomu systemu}
    \acceptcrit{Zakładka „Groups” zaimplementowana w systemie}
    \inputdata{\gls{użytkownik} systemu z przypisaną rolą administratora}
    \preconditions{\gls{użytkownik} zalogowany do systemu}
    \postconditions{\gls{użytkownik} widzi zakładkę „Groups” na pasku bocznym, po której kliknięciu przeniesiony zostaje na widok grup \gls{użytkownik}ów, widoczny jest przycisk „Add Group”, po kliknięciu w który wyświetla się formularz tworzenia grupy}
    \exceptions{Brak}
    \implementation{Dodanie zakładki „Groups” na pasku bocznym, implementacja formularza dodania grupy}
    \sholder{Zespół projektowy}
    \reqrelated{LAB-F-18}
\end{requirementstab}

\subsection{Zakładka "Admin"}

% LAB-F-20
\begin{requirementstab}
    \id{LAB-F-20}
    \priority{M}
    \name{System musi zezwalać na tworzenie nowych \gls{użytkownik}ów}
    \descr{Jako \gls{użytkownik} systemu z rolą administratora chcę mieć możliwość tworzenia nowych \gls{użytkownik}ów}
    \acceptcrit{Funkcjonalność tworzenia \gls{użytkownik}ów i zakładka „Admin” dostępna w systemie}
    \inputdata{\gls{użytkownik} z rolą administratora}
    \preconditions{\gls{użytkownik} zalogowany do systemu widzi zakładkę „Admin” na pasku bocznym}
    \postconditions{Po kliknięciu w zakładkę „Admin” \gls{użytkownik} przeniesiony jest do panelu, w którym ma możliwość utworzenia nowego \gls{użytkownik}a}
    \exceptions{Brak}
    \implementation{Dodanie zakładki „Admin” i implementacja logiki tworzenia nowego \gls{użytkownik}a}
    \sholder{Zespół projektowy}
    \reqrelated{Brak}
\end{requirementstab}

\subsection{Imort \& Export}

% LAB-F-13
\begin{requirementstab}
    \id{LAB-F-13}
    \priority{C}
    \name{System powinien pozwalać \gls{użytkownik}om na wybór między zastąpieniem i utworzeniem kopii danych importowanych z zewnętrznych źródeł}
    \descr{Jak \gls{użytkownik} systemu chcę mieć możliwość importu i eksportu danych z systemu do/z pliku}
    \acceptcrit{System posiada funkcjonalny mechanizm importu i eksportu danych}
    \inputdata{\gls{użytkownik} systemu z rolą administratora}
    \preconditions{\gls{użytkownik} zalogowany do systemu, widzi zakładkę „Import \& Export” na pasku bocznym}
    \postconditions{Po kliknięciu w zakładkę „Import \& Export” \gls{użytkownik} przenoszony jest do formularza z możliwością importu i exportu danych}
    \exceptions{Brak miejsca na dysku, plik w niewłaściwym formacie}
    \implementation{Dodanie zakładki „Import \& Export” do paska bocznego, implementacja logiki exportu i importu danych}
    \sholder{Zespół projektowy}
    \reqrelated{Brak}
\end{requirementstab}

%----------------------------------------------------------------------------------------------------
%----------------------------------------INTERFESJ-Z-OTOCZENIEM--------------------------------------
%----------------------------------------------------------------------------------------------------

\section{Interfejs z otoczeniem}

% LAB-I-01
\begin{requirementstab}
    \id{LAB-I-01}
    \priority{M}
    \name{System musi zezwalać na import i eksport danych z arkuszy kalkulacyjnych}
    \descr{Jako \gls{użytkownik} systemu chcę mieć możliwość importu i eksportu danych do/z plików arkuszy kalkulacyjnych}
    \acceptcrit{Zaimplementowana logika importu z arkuszy kalkulacyjnych}
    \inputdata{\gls{użytkownik} systemu z rolą administratora, zaimplementowana zakładka „Import \& Export”}
    \preconditions{\gls{użytkownik} zalogowany do systemu na ekranie formularza służącego do importu i exportu danych}
    \postconditions{\gls{użytkownik} jest w stanie importować oraz eksportować dane do arkusza kalkulacyjnego}
    \exceptions{Brak}
    \implementation{Implementacja logiki importu i eksportu do arkuszy kalkulacyjnych}
    \sholder{Zespół projektowy}
    \reqrelated{LAB-F-13}
\end{requirementstab}

% LAB-I-02
\begin{requirementstab}
    \id{LAB-I-02}
    \priority{C}
    \name{System powinien zezwalać na integrację z systemami \gls{PDU}}
    \descr{Jako \gls{użytkownik} systemu chcę mieć dostęp do systemów \gls{PDU} z poziomu systemu }
    \acceptcrit{Zaimplementowana obsługa systemów \gls{PDU}}
    \inputdata{\gls{użytkownik} systemu z dowolną rolą, przynajmniej 1 urządzenie w systemie fizycznie podłączone do \gls{PDU}}
    \preconditions{\gls{użytkownik} zalogowany do systemu, na ekranie szczegółów urządzenia widzi panel związany z systemem \gls{PDU}}
    \postconditions{\gls{użytkownik} jest w stanie wykonywać akcję na systemie \gls{PDU} korzystając z widoku szczegółów urządzenia}
    \exceptions{Brak}
    \implementation{Implementacja integracji z systemami \gls{PDU} na poziomie szczegółów urządzenia}
    \sholder{Zespół projektowy}
    \reqrelated{Brak}
\end{requirementstab}

% LAB-I-03
\begin{requirementstab}
    \id{LAB-I-03}
    \priority{M}
    \name{System musi zezwalać na import danych z agentów systemu \gls{Prometheus}}
    \descr{Jako \gls{użytkownik} systemu chcę widzieć dane z agentów systemu \gls{Prometheus} na ekranie szczegółów urządzenia}
    \acceptcrit{Zaimplementowana obsługa importu danych z agentów systemu \gls{Prometheus}}
    \inputdata{\gls{użytkownik} systemu z dowolną rolą, przynajmniej 1 urządzenie posiadające zainstalowany system \gls{Prometheus}}
    \preconditions{\gls{użytkownik} zalogowany do systemu, na poziomie widoku szczegółów urządzenia }
    \postconditions{\gls{użytkownik} widzi dane zaimportowane z systemu \gls{Prometheus}}
    \exceptions{Brak}
    \implementation{Implementacja importu danych do systemu z agentów systemu \gls{Prometheus}}
    \sholder{Zespół projektowy}
    \reqrelated{Brak}
\end{requirementstab}

% LAB-I-04
\begin{requirementstab}
    \id{LAB-I-04}
    \priority{M}
    \name{System musi zezwalać na import danych z \gls{Ansible}}
    \descr{Jako \gls{użytkownik} chcę mieć dostęp do danych z \gls{Ansible}}
    \acceptcrit{Zaimplementowana logika importu danych z \gls{Ansible}}
    \inputdata{Przynajmniej 1 urządzenie dostępne w sieci lokalnej i w systemie \gls{Labbyn}}
    \preconditions{Wysłanie żądania używając API \gls{Ansible}}
    \postconditions{Żądanie przetworzone poprawnie, akcje zapisane w \gls{Playbook}’u \gls{Ansible} wykonane pomyślnie, dane zaimportowane do systemu}
    \exceptions{Zerwanie połączenia sieci lokalnej}
    \implementation{Zaimplementowanie obsługi przetwarzania żądań \gls{Ansible} i importu danych do systemu}
    \sholder{Zespół projektowy}
    \reqrelated{Brak}
\end{requirementstab}

%----------------------------------------------------------------------------------------------------
%----------------------------------------NIEFUNKCJONALNE---------------------------------------------
%----------------------------------------------------------------------------------------------------

\section{Wymagania pozafunkcjonalne}

% LAB-NF-01
\begin{requirementstab}
    \id{LAB-NF-01}
    \priority{S}
    \name{Przetwarzanie danych systemu}
    \descr{System powinien umożliwiać sprawne wprowadzanie dużych ilości masowych danych}
    \acceptcrit{Odpowiednio zoptymalizowany system obsługi danych}
    \sholder{Zespół projektowy}
    \reqrelated{Brak}
\end{requirementstab}

% LAB-NF-02
\begin{requirementstab}
    \id{LAB-NF-02}
    \priority{S}
    \name{Jednoczesne połączenia}
    \descr{System powinien umożliwiać jednoczesne połączenie wielu \gls{użytkownik}ów. Wartość maksymalna opisana jest w pliku konfiguracyjnym jako parametr "maxClientsNumber"}
    \acceptcrit{Implementacja systemu wielu połączeń do systemu jednocześnie}
    \sholder{Zespół projektowy}
    \reqrelated{Brak}
\end{requirementstab}

% LAB-NF-03
\begin{requirementstab}
    \id{LAB-NF-03}
    \priority{M}
    \name{Autoryzacja \gls{użytkownik}ów}
    \descr{System musi autoryzować wykonywane w nim akcje na podstawie roli/uprawnień \gls{użytkownik}a}
    \acceptcrit{Zaimplementowany mechanizm autoryzacji \gls{użytkownik}ów}
    \sholder{Zespół projektowy}
    \reqrelated{Brak}
\end{requirementstab}

% LAB-NF-04
\begin{requirementstab}
    \id{LAB-NF-04}
    \priority{S}
    \name{Przenaszalność systemu}
    \descr{System powinien pozwolić na przenoszenie go między różnymi środowiskami}
    \acceptcrit{Implementacja systemu w formie łatwej do przeniesienia}
    \sholder{Zespół projektowy}
    \reqrelated{Brak}
\end{requirementstab}

% LAB-NF-05
\begin{requirementstab}
    \id{LAB-NF-05}
    \priority{S}
    \name{Utrzymanie i aktualizacja}
    \descr{System powinien być łatwy w utrzymaniu oraz aktualizacji}
    \acceptcrit{Implementacja mechanizmu aktualizacji, uproszczenie i minimalizacja zależności systemu}
    \sholder{Zespół projektowy}
    \reqrelated{Brak}
\end{requirementstab}

% LAB-NF-06
\begin{requirementstab}
    \id{LAB-NF-06}
    \priority{S}
    \name{Doświadczenia \gls{użytkownik}a}
    \descr{System powinien zapewnić standardy i dobre praktyki UX}
    \acceptcrit{Implementacja systemu zgodnie z przyjętymi standardami UX}
    \sholder{Zespół projektowy}
    \reqrelated{Brak}
\end{requirementstab}

% LAB-NF-07
\begin{requirementstab}
    \id{LAB-NF-07}
    \priority{M}
    \name{Domyślny \gls{użytkownik}}
    \descr{System musi posiadać wbudowanego \gls{użytkownik}a serwis}
    \acceptcrit{Implementacja autoryzacji pierwszego uruchomienia}
    \sholder{Zespół projektowy}
    \reqrelated{Brak}
\end{requirementstab}

% LAB-NF-08
\begin{requirementstab}
    \id{LAB-NF-08}
    \priority{M}
    \name{Przechowywanie haseł \gls{użytkownik}ów}
    \descr{System musi przechowywać hasła \gls{użytkownik}ów w formie zaszyfrowanej żeby zapewnić bezpieczeństwo danych}
    \acceptcrit{Zaimplementowany mechanizm szyfrowania haseł po stronie systemu}
    \sholder{Zespół projektowy}
    \reqrelated{Brak}
\end{requirementstab}

%----------------------------------------------------------------------------------------------------
%----------------------------------------ŚRODOWISKO DOCELOWE-----------------------------------------
%----------------------------------------------------------------------------------------------------

\section{Wymagania na środowisko docelowe}

% LAB-E-01
\begin{requirementstab}
    \id{LAB-E-01}
    \priority{M}
    \name{Wspierane przeglądarki}
    \descr{System musi być kompatybilny z przeglądarkami na silniku Chromium}
    \acceptcrit{W pełni działający system dostępny w przeglądarkach na silniku Chromium}
    \sholder{Zespół projektowy}
    \reqrelated{Brak}
\end{requirementstab}

% LAB-E-02
\begin{requirementstab}
    \id{LAB-E-02}
    \priority{S}
    \name{System operacyjny}
    \descr{System powinien działać i być kompatybilny z systemami Linux}
    \acceptcrit{W pełni działający system \gls{Labbyn} na systemach Linux}
    \sholder{Zespół projektowy}
    \reqrelated{Brak}
\end{requirementstab}

% LAB-E-03
\begin{requirementstab}
    \id{LAB-E-03}
    \priority{M}
    \name{Środowisko domyślne}
    \descr{System musi działać w środowisku kontenerowym}
    \acceptcrit{Implementacja systemu wspierającego środowisko kontenerowe}
    \sholder{Zespół projektowy}
    \reqrelated{Brak}
\end{requirementstab}
