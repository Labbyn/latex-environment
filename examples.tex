\chapter{Przykładowe elementy}

W niniejszym rozdziale pokażę różne ciekawe elementy które można użyć w pracy. Nie są to wszystkie możliwości, ale te, które moim zdaniem najczęściej pojawiają się w pracach inż. Na przykład, jeśli chcemy powołać się na jakieś źródła, robimy to tak~\cite{BEHESHTIROUI2021107419}.

\section{Listingi}

Jak załączać kod źródłowy jest pokazane na listingu~\ref{lst:helloworld}


\begin{lstlisting}[language=c,caption={Przykładowy witaj w świecie}, label={lst:helloworld}]
printf("hello");
\end{lstlisting}

\section{Obrazki}

Na ilustracji~\ref{img:pjatklogo} widzimy oficjalne logo PJATK.

\begin{figure}[h!]
    \centering
    \includegraphics[width=0.5\textwidth]{images/pjatk}
    \caption{Logo PJATK załączone jako obrazek}
    \label{img:pjatklogo}
\end{figure}

albo dla wygody jako makro tak jak na obrazku~\ref{img:pjatklogo2}

\putimage{Obrazek załączony za pomocą makra}{images/pjatk}{img:pjatklogo2}{0.5\textwidth}