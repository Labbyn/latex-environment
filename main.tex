\documentclass{sprz}

\addbibresource{bibliography.bib}

\studfield{Informatyka}
\studtype{Zaoczne}
\title{System do zarządzania infrastrukturą sprzętową}
\engtitle{Hardware infrastructure management system}
\acronym{Labbyn}
\titledate{2025-10-26}
\supervisor{mgr inż. Paweł Pisarski}
\reviewer{BRAK}
\author{Adrian Kopczyński}{s26990}{Cyberbezpieczeństwo}{Niestacjonarny}
\author{Patryk Kośmider}{s26985}{Sztuczna Inteligencja}{Niestacjonarny}
\author{Ziemowit Orlikowski}{s27081}{Sztuczna Inteligencja}{Niestacjonarny}
\author{Aleksander Stankowski}{s27549}{Cyberbezpieczeństwo}{Niestacjonarny}
\consultant{--- brak ---} % Koniecznie trzeba podać brak, albo wpisać konsultantów tak jak przy autorach
\projectgoals{Stworzenie systemu zarządzania infrastrukturą sprzętową który umożliwia łatwy  dostęp do statystyk, lokalizacji i właściwości sprzętu}
\productsandservices{W pełni funkcjonalny system zarządzania infrastrukturą sprzętową w laboratorium sprzętowym lub serwerowni}
\mainfunctionalities{}
\successmeasure{Systematyczność pracy zespołu: Regularne cotygodniowe przeglądy postępów prac na spotkaniach zespołu,
Implementacja wszystkich wymagań oznaczonych jako “must”,
}
\projlimitations{Projekt musi zostać ukończony przed datą obrony pracy, co stanowi kluczowy termin zakończenia prac,
Ograniczona liczba programistów do implementacji,
Ograniczenia finansowe; Brak dostępu do płatnych/bardziej zaawansowanych narzędzi,
}
\date{Lipiec, 2026}
\finishdate{\today}
\nabstract{To jest streszczenie niezwykle innowacyjnej pracy inżynierskiej w której przeplatają się elementy paplaniny z wygenerowanym tekstem pseudonaukowym rozpoznawanym jako dezinformacja.}

\keyword{komputer}


\begin{document}

\maketitle

\makeprojectcard

\tableofcontents

\chapter{Informacje wstępne}



\section{O projekcie}\label{ch:wstep}

To jest przykładowy szablon pracy inżynierskiej na PJATK.

\chapter{Przykładowe elementy}

W niniejszym rozdziale pokażę różne ciekawe elementy które można użyć w pracy. Nie są to wszystkie możliwości, ale te, które moim zdaniem najczęściej pojawiają się w pracach inż. Na przykład, jeśli chcemy powołać się na jakieś źródła, robimy to tak~\cite{BEHESHTIROUI2021107419}.

\section{Listingi}

Jak załączać kod źródłowy jest pokazane na listingu~\ref{lst:helloworld}


\begin{lstlisting}[language=c,caption={Przykładowy witaj w świecie}, label={lst:helloworld}]
printf("hello");
\end{lstlisting}

\section{Obrazki}

Na ilustracji~\ref{img:pjatklogo} widzimy oficjalne logo PJATK.

\begin{figure}[h!]
    \centering
    \includegraphics[width=0.5\textwidth]{images/pjatk}
    \caption{Logo PJATK załączone jako obrazek}
    \label{img:pjatklogo}
\end{figure}

albo dla wygody jako makro tak jak na obrazku~\ref{img:pjatklogo2}

\putimage{Obrazek załączony za pomocą makra}{images/pjatk}{img:pjatklogo2}{0.5\textwidth}

\chapter{Udziałowcy}


\begin{stakeholder}[label={tab:stakeholder:someholder},caption={Udziałowiec pierwszy}]
    \id{UOB\_01}
    \name{Zespół projektowy}
    \descr{Zespół projektowy systemu \gls{Labbyn} tworzący oprogramowanie}
    \type{Ożywniony, bezpośredni}
    \viewpoint{Techniczny, operator systemu}
    \limitations{Ograniczenia czasowe, ograniczenia budżetowe}
    \requ{LAB-D-01, LAB-D-02, LAB-D-03, LAB-D-04, LAB-D-05, LAB-D-06, LAB-E-01, LAB-E-02, LAB-E-03,
LAB-F-01, LAB-F-02, LAB-F-03, LAB-F-04, LAB-F-05, LAB-F-06, LAB-F-07, LAB-F-08, LAB-F-09, LAB-
F-10, LAB-F-11, LAB-F-12, LAB-F-13, LAB-F-14, LAB-F-15, LAB-F-16, LAB-F-17, LAB-F-18, LAB-F-19,
LAB-F-20, LAB-F-21, LAB-I-01, LAB-I-02, LAB-I-03, LAB-I-04, LAB-NF-01, LAB-NF-02, LAB-NF-03,
LAB-NF-04, LAB-NF-05, LAB-NF-06, LAB-NF-07, LAB-NF-08}
\end{stakeholder}

\begin{stakeholder}[label={tab:stakeholder:someholder},caption={Udziałowiec drugi}]
    \id{UOB\_02}
    \name{Opiekun projektu}
    \descr{Recenzent projektu inżynieryjnego}
    \type{Ożywniony, bezpośredni}
    \viewpoint{Biznesowy}
    \limitations{Brak wpływu na tworzony kod}
    \requ{Brak}
\end{stakeholder}

\chapter{Wymagania na system}

%----------------------------------------------------------------------------------------------------
%---------------------------------------OGÓLNE-I-DZIEDZINOWE-----------------------------------------
%----------------------------------------------------------------------------------------------------

\section{Wymagania ogólne i dziedzinowe}
% LAB-D-01
\begin{requirementstab}
    \id{LAB-D-01}
    \priority{M}
    \name{Kontrola dostępu}
    \descr{System musi być dostępny tylko dla użytkowników posiadających konto w systemie}
    \sholder{Zespół projektowy}
    \reqrelated{Brak}
\end{requirementstab}

% LAB-D-02
\begin{requirementstab}
    \id{LAB-D-02}
    \priority{M}
    \name{Zarządzanie Laboratorium}
    \descr{System musi wspierać zarządzanie laboratorium IT}
    \sholder{Zespół projektowy}
    \reqrelated{Brak}
\end{requirementstab}

% LAB-D-03
\begin{requirementstab}
    \id{LAB-D-03}
    \priority{M}
    \name{Inwentarz Techniczny}
    \descr{System musi wspierać zarządzanie inwentarzem technicznym}
    \sholder{Zespół projektowy}
    \reqrelated{Brak}
\end{requirementstab}

% LAB-D-04
\begin{requirementstab}
    \id{LAB-D-04}
    \priority{M}
    \name{Monitorowanie stanu}
    \descr{System musi wspierać monitorowanie stanu urządzeń}
    \sholder{Zespół projektowy}
    \reqrelated{Brak}
\end{requirementstab}

% LAB-D-05
\begin{requirementstab}
    \id{LAB-D-05}
    \priority{M}
    \name{Mapa przestrzenna laboratorium}
    \descr{System musi umożliwiać odwzorowanie fizycznego układu laboratorium}
    \sholder{Zespół projektowy}
    \reqrelated{Brak}
\end{requirementstab}

% LAB-D-06
\begin{requirementstab}
    \id{LAB-D-06}
    \priority{M}
    \name{Ochrona danych}
    \descr{System musi spełniać podstawowe wymogi ochrony danych}
    \sholder{Zespół projektowy}
    \reqrelated{Brak}
\end{requirementstab}

%----------------------------------------------------------------------------------------------------
%----------------------------------------FUNKCJONALNE------------------------------------------------
%----------------------------------------------------------------------------------------------------

\section{Wymagania Funkcjonalne}

\subsection{Zakładka "Labs"}

% LAB-F-01
\begin{requirementstab}
    \id{LAB-F-01}
    \priority{M}
    \name{System musi posiadać widok całego laboratorium}
    \descr{Jako pracownik związany ze sprzętem w laboratorium/serwerowni chcę mieć możliwość
sprawdzenia układu pomieszczenia w celu efektownego planowania przyszłych i bieżących zadań}
    \acceptcrit{System posiada w pełni funkcjonalną zakładkę „Map” dla każdego laboratorium}
    \inputdata{Lab z mapą przestrzenną, użytkownik systemu z dowolną przypisaną rolą}
    \preconditions{Użytkownik jest zalogowany do systemu i widzi zakładkę „Labs”, a wewnątrz niej przynajmniej 1 Lab}
    \postconditions{Mapa przestrzenna laboratorium jest widoczna i w pełni funkcjonalna}
    \exceptions{Lab nie posiada mapy laboratorium}
    \implementation{Dodanie przycisku „Show Map” do każdego obiektu laboratorium.}
    \sholder{Zespół projektowy}
    \reqrelated{LAB-F-14}
\end{requirementstab}

% LAB-F-14
\begin{requirementstab}
    \id{LAB-F-14}
    \priority{M}
    \name{System musi posiadać system Drag n' Drop elementów laboratorium}
    \descr{Jako użytkownik systemu, chcę móc ustawić elementy interaktywne na mapie przestrzennej laboratorium, żeby odwzorować stan faktyczny pomieszczenia.}
    \acceptcrit{Mapa przestrzenna laboratorium posiada w pełni funkcjonalny system Drag n’ Drop z możliwością aranżacji elementów na dostępnej powierzchni.}
    \inputdata{Lab z mapą przestrzenną, użytkownik systemu z dowolną przypisaną rolą z uprawnieniami do wybranego laboratorium}
    \preconditions{Użytkownik jest zalogowany do systemu, w zakładce „Labs” widzi przynajmniej 1 lab do którego ma uprawnienia edycji z przyciskiem „Show Map”, po kliknięciu na ten przycisk przeniesiony jest do widoku przestrzennego laboratorium.}
    \postconditions{Użytkownik jest w stanie rozmieszczać elementy laboratorium w dowolny sposób z możliwością zapisu zmian.}
    \exceptions{Użytkownik nie posiada praw edycji laboratorium na mapie przestrzennej.}
    \implementation{Dodanie przycisku „Edit” możliwego do kliknięcia w momencie gdy zalogowany użytkownik jest przypisany do danego laboratorium.}
    \sholder{Zespół projektowy}
    \reqrelated{LAB-F-01}
\end{requirementstab}

\subsection{Panel Urządzenia}

% LAB-F-02
\begin{requirementstab}
    \id{LAB-F-02}
    \priority{M}
    \name{System musi pokazywać statystyki urządzeń}
    \descr{Jako użytkownik systemu, chcę mieć możliwość odczytania statystyk każdego urządzenia w jego unikalnej zakładce z informacjami}
    \acceptcrit{Każde urządzenie posiada widok ze szczegółowymi statystykami}
    \inputdata{System z przynajmniej 1 dostępnym urządzeniem, użytkownik systemu z dowolną przypisaną rolą}
    \preconditions{Użytkownik jest zalogowany do systemu, w zakładce „Labs” widzi przynajmniej 1 laboratorium z przypisanym przynajmniej 1 urządzeniem, który może kliknąć}
    \postconditions{Użytkownik jest w stanie przenieść się na widok ze szczegółami urządzenia po kliknięciu kafelka z nazwą urządzenia}
    \exceptions{Brak dostępnych urządzeń w systemie}
    \implementation{Po kliknięciu na obiekt maszyny, aktywowanie okna urządzenia ze statystykami}
    \sholder{Zespół projektowy}
    \reqrelated{LAB-F-03, LAB-F-05, LAB-F-10, LAB-F-11, LAB-F-15}
\end{requirementstab}

% LAB-F-03
\begin{requirementstab}
    \id{LAB-F-03}
    \priority{C}
    \name{System powinien pozwolić na wykonywanie zdalnych akcji np. Reboot, ping}
    \descr{Jak użytkownik systemu chcę mieć dostęp do wykonywania zdalnych komend na maszynie z poziomu widoku maszyny}
    \acceptcrit{System posiada możliwość wykonania zdalnej komendy z poziomu interfejsu użytkownika}
    \inputdata{System z przynajmniej 1 dostępnym urządzeniem, użytkownik systemu z dowolną przypisaną rolą}
    \preconditions{Użytkownik jest zalogowany do systemu, na ekranie szczegółów urządzenia widoczny jest panel do wpisania zdalnej komendy}
    \postconditions{Użytkownik jest w stanie wysłać komendę zdalnie na urządzenie z poziomu systemu}
    \exceptions{Wybrane urządzenie nie jest podłączone do sieci}
    \implementation{Na ekranie szczegółów urządzenia dodanie sekcji do wysyłania zdalnego polecenia.}
    \sholder{Zespół projektowy}
    \reqrelated{LAB-F-02, LAB-F-05, LAB-F-10, LAB-F-11, LAB-F-15}
\end{requirementstab}

% LAB-F-05
\begin{requirementstab}
    \id{LAB-F-05}
    \priority{M}
    \name{System musi zezwolić na przypisywanie platform do konkretnych racków i półek}
    \descr{Jako użytkownik chcę mieć możliwość przypisania platformy do wybranej przeze mnie półki w celu skutecznego wyszukiwania urządzeń znajdujących się na tej samej półce lub racku.}
    \acceptcrit{System posiada możliwość przypisania platformy do jednego z racków lub półek.}
    \inputdata{System z przynajmniej 1 dostępnym urządzeniem, labem oraz półką lub rackiem, użytkownik systemu z dowolną przypisaną rolą z przynajmniej 1 przypisaną do niego lub jego grupy maszyną}
    \preconditions{Użytkownik jest zalogowany do systemu, na ekranie szczegółów urządzenia widoczny jest aktualnie przypisany Rack lub półka}
    \postconditions{Użytkownik jest w stanie zmienić przypisanie urządzenia do innego recku lub półki}
    \exceptions{Brak innych racków lub półek w systemie}
    \implementation{Na ekranie szczegółów urządzenia dodanie sekcji z przypisanym rackiem lub półką}
    \sholder{Zespół projektowy}
    \reqrelated{LAB-F-02, LAB-F-03, LAB-F-10, LAB-F-11, LAB-F-15}
\end{requirementstab}

% LAB-F-10
\begin{requirementstab}
    \id{LAB-F-10}
    \priority{W}
    \name{System powinien posiadać odnośnik do informacji o danym sprzęcie/ layout'u PCB}
    \descr{Jako użytkownik chcę mieć możliwość przejść do karty produktu ze szczegółami technicznymi lub dotyczącymi PCB z poziomu systemu}
    \acceptcrit{System posiada odnośnik do strony z dokumentacją techniczną sprzętu}
    \inputdata{System z przynajmniej 1 dostępnym urządzeniem, użytkownik systemu z dowolną przypisaną rolą}
    \preconditions{Brak, wymaganie odrzucone}
    \postconditions{Brak, wymaganie odrzucone}
    \exceptions{Brak, wymaganie odrzucone}
    \implementation{Brak, wymaganie odrzucone}
    \sholder{Zespół projektowy}
    \reqrelated{LAB-F-02, LAB-F-03, LAB-F-05, LAB-F-11, LAB-F-15}
\end{requirementstab}

% LAB-F-11
\begin{requirementstab}
    \id{LAB-F-11}
    \priority{M}
    \name{Dane w widoku urządzenia}
    \descr{Jako użytkownik systemu, chcę mieć dostęp do następujących danych na poziomie widoku urządzenia: Lokalizacja, Nazwa Hosta, Adres IP oraz MAC, Status w sieci, Port w PDU, Tag (przypisany zespół / przeznaczony zespół), System Operacyjny, Informacje o sprzęcie (CPU, Dysk, RAM), Numer Seryjny}
    \acceptcrit{Wszystkie pola zawarte w wymaganiu są dostępne w systemie w widoku urządzenia}
    \inputdata{System z przynajmniej 1 dostępnym urządzeniem, użytkownik systemu z dowolną przypisaną rolą z przynajmniej 1 przypisaną do niego lub jego grupy maszyną}
    \preconditions{Użytkownik jest zalogowany do systemu, na ekranie szczegółów urządzenia}
    \postconditions{Użytkownik jest w stanie zobaczyć wszystkie dane lub miejsce na ich umieszczenie w systemie na poziomie widoku urządzenia}
    \exceptions{Brak dostępnych lub przypisanych do użytkownika urządzeń w systemie}
    \implementation{Na ekranie szczegółów urządzenia dodać pola wypisane w wymaganiu.}
    \sholder{Zespół projektowy}
    \reqrelated{LAB-F-02, LAB-F-03, LAB-F-05, LAB-F-10, LAB-F-15}
\end{requirementstab}

% LAB-F-15
\begin{requirementstab}
    \id{LAB-F-15}
    \priority{M}
    \name{System musi posiadać widok urządzenia}
    \descr{Jako użytkownik systemu chcę mieć dostęp do szczegółów urządzenia z poziomu systemu}
    \acceptcrit{Zaimplementowany panel szczegółów urządzenia}
    \inputdata{System z przynajmniej 1 dostępnym urządzeniem, użytkownik systemu z dowolną przypisaną rolą z przynajmniej 1 przypisaną do niego lub jego grupy maszyną}
    \preconditions{Użytkownik jest zalogowany do systemu}
    \postconditions{Użytkownik jest w stanie przejść do widoku ze szczegółami urządzenia z poziomu systemu}
    \exceptions{Brak dostępnych urządzeń w systemie}
    \implementation{Dodanie widoku urządzenia po kliknięciu w kafelek z nazwą urządzenia z poziomu mapy przestrzennej lub paska wyszukiwania}
    \sholder{Zespół projektowy}
    \reqrelated{LAB-F-02, LAB-F-03, LAB-F-05, LAB-F-10, LAB-F-11}
\end{requirementstab}

\subsection{Zakładka "Inventory"}

% LAB-F-04
\begin{requirementstab}
    \id{LAB-F-04}
    \priority{S}
    \name{System powinien posiadać inwentarz materiałów eksploatacyjnych}
    \descr{Jako użytkownik systemu chcę mieć dostęp do inwentarza materiałów eksploatacyjnych aby szybko i efektywnie wyszukiwać niezbędne do mojej pracy narzędzia i kontrolować ich ilość}
    \acceptcrit{Zakładka „Inventory” zaimplementowana w systemie}
    \inputdata{Użytkownik systemu z dowolną przypisaną rolą}
    \preconditions{Użytkownik jest zalogowany do systemu}
    \postconditions{Użytkownik widzi w pasku bocznym zakładkę „Inventory”, po której kliknięciu przenoszony jest na ekran inwentarza}
    \exceptions{Brak paska bocznego}
    \implementation{Dodanie zakładki „Inventory” w paski bocznym}
    \sholder{Zespół projektowy}
    \reqrelated{LAB-F-12}
\end{requirementstab}

% LAB-F-12
\begin{requirementstab}
    \id{LAB-F-12}
    \priority{M}
    \name{Informacje w inwentarzu technicznym}
    \descr{Jako użytkownik systemu, chcę mieć dostęp do następujących danych na poziomie inwentarza technicznego: Nazwa i nr seryjny, Dostępność, przypisanie do konkretnych maszyn, przypisanie do konkretnych zespołów, typ urządzenia/kategorie, Historia wypożyczeń, lokalizacja}
    \acceptcrit{Wszystkie pola zawarte w wymaganiu są dostępne w systemie w inwentarzu technicznym}
    \inputdata{Użytkownik systemu z dowolną przypisaną rolą, przynajmniej 1 element w inwentarzu przypisany do użytkownika lub jego grupy}
    \preconditions{Użytkownik zalogowany do systemu na zakładce „Inventory”}
    \postconditions{Użytkownik jest w stanie zobaczyć wszystkie dane lub miejsce na ich umieszczenie w systemie na poziomie inwentarzu}
    \exceptions{Brak przypisanych elementów do użytkownika w inwentarzu}
    \implementation{Do każdego elementu inwentarzu dodać pola wypisane w wymaganiu.}
    \sholder{Zespół projektowy}
    \reqrelated{LAB-F-04}
\end{requirementstab}

\subsection{Zakładka "Dashboard"}

% LAB-F-06
\begin{requirementstab}
    \id{LAB-F-06}
    \priority{S}
    \name{System powinien posiadać panel z podsumowaniem informacji o sprzęcie przypisanym do użytkownika}
    \descr{Jako użytkownik systemu chcę mieć dostęp do panelu z podsumowaniem elementów przypisanych do mnie, co pozwoli skrócić czas wyszukiwania go}
    \acceptcrit{Zakładka „Dashboard” widoczna z poziomu systemu na pasku bocznym}
    \inputdata{Użytkownik systemu z dowolną przypisaną rolą}
    \preconditions{Użytkownik zalogowany do systemu}
    \postconditions{Użytkownik widzi zakładkę „Dashboard” na pasku bocznym, po której kliknięciu przenoszony jest na ekran z podsumowaniem}
    \exceptions{Użytkownik nie ma przypisanego żadnego elementu}
    \implementation{Dodanie zakładki „Dashboard” na pasku bocznym}
    \sholder{Zespół projektowy}
    \reqrelated{Brak}
\end{requirementstab}

\subsection{Search Bar (Pasek wyszukiwania)}

% LAB-F-07
\begin{requirementstab}
    \id{LAB-F-07}
    \priority{M}
    \name{System musi posiadać wyszukiwarkę sprzętu}
    \descr{Jak użytkownik systemu chcę mieć dostęp do paska wyszukiwania by przyśpieszyć identyfikowanie sprzętu}
    \acceptcrit{Pasek wyszukiwania zaimplementowany w systemie i dodany do paska bocznego}
    \inputdata{Użytkownik systemu z dowolną przypisaną rolą}
    \preconditions{Użytkownik zalogowany do systemu}
    \postconditions{Użytkownik widzi pasek wyszukiwania na pasku bocznym, w którym jest w stanie wpisać wyszukiwaną frazę}
    \exceptions{Brak}
    \implementation{Dodanie paska wyszukiwania na pasku bocznym}
    \sholder{Zespół projektowy}
    \reqrelated{Brak}
\end{requirementstab}

\subsection{Zakładka "History"}

% LAB-F-08
\begin{requirementstab}
    \id{LAB-F-08}
    \priority{M}
    \name{System musi posiadać historię zmian sprzętu}
    \descr{Jako użytkownik chcę śledzić zmiany sprzętu by efektywnie identyfikować modyfikacje w jego konfiguracji}
    \acceptcrit{Mechanizm zapisywania historii akcji zaimplementowany w systemie}
    \inputdata{Użytkownik systemu z dowolną przypisaną rolą i przynajmniej 1 przypisanym elementem do niego lub jego grupy}
    \preconditions{Użytkownik zalogowany do systemu na ekranie wybranego elementu}
    \postconditions{Użytkownik wykonuje zmianę, która jest wykryta przez system}
    \exceptions{Brak}
    \implementation{Dodanie mechanizmu zapisywania wykonywanych akcji w systemie}
    \sholder{Zespół projektowy}
    \reqrelated{LAB-F-16, LAB-F-21}
\end{requirementstab}

% LAB-F-16
\begin{requirementstab}
    \id{LAB-F-16}
    \priority{C}
    \name{System powinien rejestrować działania użytkowników i zapisywać je do pliku}
    \descr{Jako użytkownik systemu chcę by zmiany wykonywane w systemie zapisywane były w pliku by zapewnić trwałość danych w kopiach zapasowych systemu}
    \acceptcrit{Zmiany logowane przez system zapisywane są do pliku}
    \inputdata{Użytkownik systemu z dowolną przypisaną rolą i przynajmniej 1 przypisanym elementem do niego lub jego grupy}
    \preconditions{Użytkownik zalogowany do systemu na ekranie wybranego elementu jest w stanie wykonać modyfikację}
    \postconditions{Zmiana wykryta przez system, zapisana do pliku}
    \exceptions{Brak miejsca na przestrzeni dyskowej, na której system domyślnie operuje}
    \implementation{Dodanie funkcjonalności zapisu historii do pliku}
    \sholder{Zespół projektowy}
    \reqrelated{LAB-F-08, LAB-F-21}
\end{requirementstab}

% LAB-F-21
\begin{requirementstab}
    \id{LAB-F-21}
    \priority{C}
    \name{Wersjonowanie historii}
    \descr{Jako użytkownik systemu chcę mieć możliwość wersjonowania historii zmian żeby w prosty sposób sprawdzić stan przed modyfikacją}
    \acceptcrit{Funkcjonalność wersjonowania historii zaimplementowana w systemie z możliwością wyboru wersji}
    \inputdata{Użytkownik systemu z dowolną przypisaną rolą i przynajmniej 1 przypisanym elementem do niego lub jego grupy, na którym wykonane zostały modyfikacje}
    \preconditions{Użytkownik z poziomu widoku wybranego elementu widzi pasek wyboru z opisem wersji}
    \postconditions{Użytkownik jest w stanie wybrać wcześniejszą wersję widoku z przed zmian}
    \exceptions{Brak zapisanych zmian w elemencie}
    \implementation{Dodanie funkcjonalności wersjonowania historii }
    \sholder{Zespół projektowy}
    \reqrelated{LAB-F-08, LAB-F-16}
\end{requirementstab}

\subsection{Zakładka "Users"}

% LAB-F-09
\begin{requirementstab}
    \id{LAB-F-09}
    \priority{M}
    \name{System musi posiadać panel zarządzania użytkownikami}
    \descr{Jako użytkownik z uprawnieniami administratora systemu chcę mieć dostęp do panelu, z poziomu którego będę w stanie zarządzać użytkownikami systemu}
    \acceptcrit{Zakładka „Users” zaimplementowana w systemie}
    \inputdata{Użytkownik systemu z przypisaną rolą administratora, dodatkowy użytkownik w systemie z dowolną rolą}
    \preconditions{Użytkownik z rolą administratora zalogowany do systemu}
    \postconditions{Użytkownik z rolą administratora widzi zakładkę „Users” na pasku bocznym systemu}
    \exceptions{Brak}
    \implementation{Dodanie i implementacja zakładki „Users” w systemie}
    \sholder{Zespół projektowy}
    \reqrelated{LAB-F-17}
\end{requirementstab}

% LAB-F-17
\begin{requirementstab}
    \id{LAB-F-17}
    \priority{M}
    \name{System powinien rozróżniać 2 role użytkowników: administrator, użytkownik}
    \descr{Jako użytkownik systemu chcę aby posiadał on kontrolę dostępu w postaci ról użytkowników: administrator oraz użytkownik}
    \acceptcrit{System posiadający role użytkowników opisane w wymaganiu}
    \inputdata{Brak}
    \preconditions{System posiadający funkcjonalność autoryzacji użytkowników}
    \postconditions{Możliwość wyboru roli pomiędzy: administrator i użytkownik}
    \exceptions{Brak}
    \implementation{Dodanie pola „role” w bazie danych użytkowników}
    \sholder{Zespół projektowy}
    \reqrelated{LAB-F-09}
\end{requirementstab}

\subsection{Zakładka "Groups"}

% LAB-F-18
\begin{requirementstab}
    \id{LAB-F-18}
    \priority{M}
    \name{System musi zezwalać na przypisywanie użytkowników do istniejących grup}
    \descr{Jako użytkownik systemu z uprawnieniami administratora chcę mieć możliwość przypisania użytkowników do grup co pozwoli efektywniej zarządzać zespołami w organizacji}
    \acceptcrit{Funkcjonalność przypisywania użytkownika zaimplementowana w systemie}
    \inputdata{System z funkcjonalną bazą danych}
    \preconditions{System z przynajmniej 1 użytkownikiem w systemie}
    \postconditions{Możliwość przypisania użytkownika do grupy}
    \exceptions{Brak}
    \implementation{Dodanie pola „group” w bazie danych użytkowników}
    \sholder{Zespół projektowy}
    \reqrelated{LAB-F-19}
\end{requirementstab}

% LAB-F-19
\begin{requirementstab}
    \id{LAB-F-19}
    \priority{M}
    \name{System musi zezwalać na tworzenie grup użytkowników}
    \descr{Jako użytkownik z uprawnieniami administratora chcę mieć możliwość tworzenia grup użytkowników z poziomu systemu}
    \acceptcrit{Zakładka „Groups” zaimplementowana w systemie}
    \inputdata{Użytkownik systemu z przypisaną rolą administratora}
    \preconditions{Użytkownik zalogowany do systemu}
    \postconditions{Użytkownik widzi zakładkę „Groups” na pasku bocznym, po której kliknięciu przeniesiony zostaje na widok grup użytkowników, widoczny jest przycisk „Add Group”, po kliknięciu w który wyświetla się formularz tworzenia grupy}
    \exceptions{Brak}
    \implementation{Dodanie zakładki „Groups” na pasku bocznym, implementacja formularza dodania grupy}
    \sholder{Zespół projektowy}
    \reqrelated{LAB-F-18}
\end{requirementstab}

\subsection{Zakładka "Admin"}

% LAB-F-20
\begin{requirementstab}
    \id{LAB-F-20}
    \priority{M}
    \name{System musi zezwalać na tworzenie nowych użytkowników}
    \descr{Jako użytkownik systemu z rolą administratora chcę mieć możliwość tworzenia nowych użytkowników}
    \acceptcrit{Funkcjonalność tworzenia użytkowników i zakładka „Admin” dostępna w systemie}
    \inputdata{Użytkownik z rolą administratora}
    \preconditions{Użytkownik zalogowany do systemu widzi zakładkę „Admin” na pasku bocznym}
    \postconditions{Po kliknięciu w zakładkę „Admin” użytkownik przeniesiony jest do panelu, w którym ma możliwość utworzenia nowego użytkownika}
    \exceptions{Brak}
    \implementation{Dodanie zakładki „Admin” i implementacja logiki tworzenia nowego użytkownika}
    \sholder{Zespół projektowy}
    \reqrelated{Brak}
\end{requirementstab}

\subsection{Imort \& Export}

% LAB-F-13
\begin{requirementstab}
    \id{LAB-F-13}
    \priority{C}
    \name{System powinien pozwalać użytkownikom na wybór między zastąpieniem i utworzeniem kopii danych importowanych z zewnętrznych źródeł}
    \descr{Jak użytkownik systemu chcę mieć możliwość importu i eksportu danych z systemu do/z pliku}
    \acceptcrit{System posiada funkcjonalny mechanizm importu i eksportu danych}
    \inputdata{Użytkownik systemu z rolą administratora}
    \preconditions{Użytkownik zalogowany do systemu, widzi zakładkę „Import \& Export” na pasku bocznym}
    \postconditions{Po kliknięciu w zakładkę „Import \& Export” użytkownik przenoszony jest do formularza z możliwością importu i exportu danych}
    \exceptions{Brak miejsca na dysku, plik w niewłaściwym formacie}
    \implementation{Dodanie zakładki „Import \& Export” do paska bocznego, implementacja logiki exportu i importu danych}
    \sholder{Zespół projektowy}
    \reqrelated{Brak}
\end{requirementstab}

%----------------------------------------------------------------------------------------------------
%----------------------------------------INTERFESJ-Z-OTOCZENIEM--------------------------------------
%----------------------------------------------------------------------------------------------------

\section{Interfejs z otoczeniem}

% LAB-I-01
\begin{requirementstab}
    \id{LAB-I-01}
    \priority{M}
    \name{System musi zezwalać na import i eksport danych z arkuszy kalkulacyjnych}
    \descr{Jako użytkownik systemu chcę mieć możliwość importu i eksportu danych do/z plików arkuszy kalkulacyjnych}
    \acceptcrit{Zaimplementowana logika importu z arkuszy kalkulacyjnych}
    \inputdata{Użytkownik systemu z rolą administratora, zaimplementowana zakładka „Import \& Export”}
    \preconditions{Użytkownik zalogowany do systemu na ekranie formularza służącego do importu i exportu danych}
    \postconditions{Użytkownik jest w stanie importować oraz eksportować dane do arkusza kalkulacyjnego}
    \exceptions{Brak}
    \implementation{Implementacja logiki importu i eksportu do arkuszy kalkulacyjnych}
    \sholder{Zespół projektowy}
    \reqrelated{LAB-F-13}
\end{requirementstab}

% LAB-I-02
\begin{requirementstab}
    \id{LAB-I-02}
    \priority{C}
    \name{System powinien zezwalać na integrację z systemami PDU}
    \descr{Jako użytkownik systemu chcę mieć dostęp do systemów PDU z poziomu systemu }
    \acceptcrit{Zaimplementowana obsługa systemów PDU}
    \inputdata{Użytkownik systemu z dowolną rolą, przynajmniej 1 urządzenie w systemie fizycznie podłączone do PDU}
    \preconditions{Użytkownik zalogowany do systemu, na ekranie szczegółów urządzenia widzi panel związany z systemem PDU}
    \postconditions{Użytkownik jest w stanie wykonywać akcję na systemie PDU korzystając z widoku szczegółów urządzenia}
    \exceptions{Brak}
    \implementation{Implementacja integracji z systemami PDU na poziomie szczegółów urządzenia}
    \sholder{Zespół projektowy}
    \reqrelated{Brak}
\end{requirementstab}

% LAB-I-03
\begin{requirementstab}
    \id{LAB-I-03}
    \priority{M}
    \name{System musi zezwalać na import danych z agentów systemu Prometheus}
    \descr{Jako użytkownik systemu chcę widzieć dane z agentów systemu Prometheus na ekranie szczegółów urządzenia}
    \acceptcrit{Zaimplementowana obsługa importu danych z agentów systemu Prometheus}
    \inputdata{Użytkownik systemu z dowolną rolą, przynajmniej 1 urządzenie posiadające zainstalowany system Prometheus}
    \preconditions{Użytkownik zalogowany do systemu, na poziomie widoku szczegółów urządzenia }
    \postconditions{Użytkownik widzi dane zaimportowane z systemu Prometheus}
    \exceptions{Brak}
    \implementation{Implementacja importu danych do systemu z agentów systemu Prometheus}
    \sholder{Zespół projektowy}
    \reqrelated{Brak}
\end{requirementstab}

% LAB-I-04
\begin{requirementstab}
    \id{LAB-I-04}
    \priority{M}
    \name{System musi zezwalać na import danych z Ansible}
    \descr{Jako użytkownik chcę mieć dostęp do danych z Ansible}
    \acceptcrit{Zaimplementowana logika importu danych z Ansible}
    \inputdata{Przynajmniej 1 urządzenie dostępne w sieci lokalnej i w systemie Labbyn}
    \preconditions{Wysłanie żądania używając API Ansible}
    \postconditions{Żądanie przetworzone poprawnie, akcje zapisane w playbook’u Ansible wykonane pomyślnie, dane zaimportowane do systemu}
    \exceptions{Zerwanie połączenia sieci lokalnej}
    \implementation{Zaimplementowanie obsługi przetwarzania żądań Ansible i importu danych do systemu}
    \sholder{Zespół projektowy}
    \reqrelated{Brak}
\end{requirementstab}

%----------------------------------------------------------------------------------------------------
%----------------------------------------NIEFUNKCJONALNE---------------------------------------------
%----------------------------------------------------------------------------------------------------

\section{Wymagania pozafunkcjonalne}

% LAB-NF-01
\begin{requirementstab}
    \id{LAB-NF-01}
    \priority{S}
    \name{Przetwarzanie danych systemu}
    \descr{System powinien umożliwiać sprawne wprowadzanie dużych ilości masowych danych}
    \acceptcrit{Odpowiednio zoptymalizowany system obsługi danych}
    \sholder{Zespół projektowy}
    \reqrelated{Brak}
\end{requirementstab}

% LAB-NF-02
\begin{requirementstab}
    \id{LAB-NF-02}
    \priority{S}
    \name{Jednoczesne połączenia}
    \descr{System powinien umożliwiać jednoczesne połączenie wielu użytkowników. Wartość maksymalna opisana jest w pliku konfiguracyjnym jako parametr "maxClientsNumber"}
    \acceptcrit{Implementacja systemu wielu połączeń do systemu jednocześnie}
    \sholder{Zespół projektowy}
    \reqrelated{Brak}
\end{requirementstab}

% LAB-NF-03
\begin{requirementstab}
    \id{LAB-NF-03}
    \priority{M}
    \name{Autoryzacja użytkowników}
    \descr{System musi autoryzować wykonywane w nim akcje na podstawie roli/uprawnień użytkownika}
    \acceptcrit{Zaimplementowany mechanizm autoryzacji użytkowników}
    \sholder{Zespół projektowy}
    \reqrelated{Brak}
\end{requirementstab}

% LAB-NF-04
\begin{requirementstab}
    \id{LAB-NF-04}
    \priority{S}
    \name{Przenaszalność systemu}
    \descr{System powinien pozwolić na przenoszenie go między różnymi środowiskami}
    \acceptcrit{Implementacja systemu w formie łatwej do przeniesienia}
    \sholder{Zespół projektowy}
    \reqrelated{Brak}
\end{requirementstab}

% LAB-NF-05
\begin{requirementstab}
    \id{LAB-NF-05}
    \priority{S}
    \name{Utrzymanie i aktualizacja}
    \descr{System powinien być łatwy w utrzymaniu oraz aktualizacji}
    \acceptcrit{Implementacja mechanizmu aktualizacji, uproszczenie i minimalizacja zależności systemu}
    \sholder{Zespół projektowy}
    \reqrelated{Brak}
\end{requirementstab}

% LAB-NF-06
\begin{requirementstab}
    \id{LAB-NF-06}
    \priority{S}
    \name{Doświadczenia użytkownika}
    \descr{System powinien zapewnić standardy i dobre praktyki UX}
    \acceptcrit{Implementacja systemu zgodnie z przyjętymi standardami UX}
    \sholder{Zespół projektowy}
    \reqrelated{Brak}
\end{requirementstab}

% LAB-NF-07
\begin{requirementstab}
    \id{LAB-NF-07}
    \priority{M}
    \name{Domyślny użytkownik}
    \descr{System musi posiadać wbudowanego użytkownika serwis}
    \acceptcrit{Implementacja autoryzacji pierwszego uruchomienia}
    \sholder{Zespół projektowy}
    \reqrelated{Brak}
\end{requirementstab}

% LAB-NF-08
\begin{requirementstab}
    \id{LAB-NF-08}
    \priority{M}
    \name{Przechowywanie haseł użytkowników}
    \descr{System musi przechowywać hasła użytkowników w formie zaszyfrowanej żeby zapewnić bezpieczeństwo danych}
    \acceptcrit{Zaimplementowany mechanizm szyfrowania haseł po stronie systemu}
    \sholder{Zespół projektowy}
    \reqrelated{Brak}
\end{requirementstab}

%----------------------------------------------------------------------------------------------------
%----------------------------------------ŚRODOWISKO DOCELOWE-----------------------------------------
%----------------------------------------------------------------------------------------------------

\section{Wymagania na środowisko docelowe}

% LAB-E-01
\begin{requirementstab}
    \id{LAB-E-01}
    \priority{M}
    \name{Wspierane przeglądarki}
    \descr{System musi być kompatybilny z przeglądarkami na silniku Chromium}
    \acceptcrit{W pełni działający system dostępny w przeglądarkach na silniku Chromium}
    \sholder{Zespół projektowy}
    \reqrelated{Brak}
\end{requirementstab}

% LAB-E-02
\begin{requirementstab}
    \id{LAB-E-02}
    \priority{S}
    \name{System operacyjny}
    \descr{System powinien działać i być kompatybilny z systemami Linux}
    \acceptcrit{W pełni działający system Labbyn na systemach Linux}
    \sholder{Zespół projektowy}
    \reqrelated{Brak}
\end{requirementstab}

% LAB-E-03
\begin{requirementstab}
    \id{LAB-E-03}
    \priority{M}
    \name{Środowisko domyślne}
    \descr{System musi działać w środowisku kontenerowym}
    \acceptcrit{Implementacja systemu wspierającego środowisko kontenerowe}
    \sholder{Zespół projektowy}
    \reqrelated{Brak}
\end{requirementstab}


\printbibliography[title={Bibliografia}, heading=bibintoc]

\makethesisattachments

\end{document}
